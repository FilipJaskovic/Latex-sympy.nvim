% latex_sympy.nvim manual feature test file
% Open this file in Neovim and run commands on the selected lines.
% Tip: use visual-line mode (V) and select one expression block at a time.

\documentclass{article}
\begin{document}

\section*{Core transforms}

% :LatexSympyEqual
x^2 + 2x + 1

% :LatexSympyReplace
\frac{d}{dx}(x^3 + x)

% :LatexSympyNumerical
\sin\left(\frac{\pi}{4}\right) + \sqrt{2}

% :LatexSympyFactor
x^2 + 2x + 1

% :LatexSympyExpand
(x + 1)^3

% :LatexSympyMatrixRREF
\begin{bmatrix} 1 & 2 \\ 3 & 4 \end{bmatrix}

\section*{Generic op command}

% :LatexSympyOp solve x
x^2 - 1 = 0

% :LatexSympyOp solve
x^2 - 4

% :LatexSympyOp solve x y
x + y - 3

% :LatexSympyOp diff x 2
x^4 + x^2

% :LatexSympyOp! diff x 2   (append mode)
x^4 + x^2

% :LatexSympyOp diff x 2 y 1
x^3 y^2

% :LatexSympyOp integrate x
x^2

% :LatexSympyOp integrate x 0 1
x^2

% :LatexSympyOp integrate x 0 1 y 0 2
x y

% :LatexSympyOp limit x 0 +-
\frac{\sin(x)}{x}

% :LatexSympyOp series x 0 6
\sin(x)

% :LatexSympyOp simplify
(x + 1)^2 - (x^2 + 2x + 1)

% :LatexSympyOp trigsimp
\sin(x)^2 + \cos(x)^2

% :LatexSympyOp ratsimp
\frac{1}{x} + \frac{1}{y}

% :LatexSympyOp powsimp
x^a x^b

% :LatexSympyOp apart x
\frac{x + 1}{x(x + 2)}

% :LatexSympyOp subs x=2 y=3
x + y

% :LatexSympyOp nsolve x 1
x^2 - 2 = 0

% :LatexSympyOp dsolve y(x)
Derivative(y(x), x) - y(x) = 0

% :LatexSympyOp solve_system x y
% (select both equations together)
x + y = 3
x - y = 1

% :LatexSympyOp solveset x R
x^2 - 1 = 0

% :LatexSympyOp linsolve x y
% (select both equations together)
x + y = 3
x - y = 1

% :LatexSympyOp nonlinsolve x y
% (select both equations together)
x^2 - 1 = 0
y - 2 = 0

% :LatexSympyOp rsolve a(n)
a(n+1) - a(n) = 0

% :LatexSympyOp diophantine x y
2x + 3y = 5

% :LatexSympyOp det
\begin{bmatrix} 1 & 2 \\ 3 & 4 \end{bmatrix}

% :LatexSympyOp inv
\begin{bmatrix} 1 & 2 \\ 3 & 4 \end{bmatrix}

% :LatexSympyOp transpose
\begin{bmatrix} 1 & 2 & 3 \\ 4 & 5 & 6 \end{bmatrix}

% :LatexSympyOp rank
\begin{bmatrix} 1 & 2 \\ 2 & 4 \end{bmatrix}

% :LatexSympyOp eigenvals
\begin{bmatrix} 1 & 2 \\ 3 & 4 \end{bmatrix}

% :LatexSympyOp eigenvects
\begin{bmatrix} 2 & 0 \\ 0 & 3 \end{bmatrix}

% :LatexSympyOp nullspace
\begin{bmatrix} 1 & 2 \\ 2 & 4 \end{bmatrix}

% :LatexSympyOp charpoly t
\begin{bmatrix} 1 & 0 \\ 0 & 2 \end{bmatrix}

% :LatexSympyOp lu
\begin{bmatrix} 4 & 3 \\ 6 & 3 \end{bmatrix}

% :LatexSympyOp qr
\begin{bmatrix} 1 & 1 \\ 1 & -1 \end{bmatrix}

% :LatexSympyOp mat_solve
\begin{bmatrix} 2 & 1 & 5 \\ 1 & -1 & 1 \end{bmatrix}

% :LatexSympyOp isprime
97

% :LatexSympyOp factorint
360

% :LatexSympyOp primerange 10 20
0

% :LatexSympyOp div x
% (select both lines together)
x^2 - 1
x - 1

% :LatexSympyOp gcd x
% (select both lines together)
x^2 - 1
x - 1

% :LatexSympyOp sqf x
(x - 1)^2 (x + 2)

% :LatexSympyOp groebner x y lex
% (select both lines together)
x^2 + y
x - y

% :LatexSympyOp resultant x
% (select both lines together)
x^2 + y
x - y

% :LatexSympyOp summation k 1 n
k

% :LatexSympyOp product k 1 n
k

% :LatexSympyOp binomial 5 2
0

% :LatexSympyOp perm 5 2
0

% :LatexSympyOp comb 5 2
0

% :LatexSympyOp partition 8
0

% :LatexSympyOp subsets 2
\left\{1, 2, 3\right\}

% :LatexSympyOp totient
36

% :LatexSympyOp mobius
30

% :LatexSympyOp divisors true
24

% :LatexSympyOp logic_simplify simplify
(A & B) | (A & ~B)

% :LatexSympyOp sat
A & ~A

% :LatexSympyOp jordan
\begin{bmatrix} 2 & 1 \\ 0 & 2 \end{bmatrix}

% :LatexSympyOp svd
\begin{bmatrix} 1 & 0 \\ 0 & 2 \end{bmatrix}

% :LatexSympyOp cholesky
\begin{bmatrix} 4 & 2 \\ 2 & 3 \end{bmatrix}

% :LatexSympyOp symbol x real=true
0

% :LatexSympyOp symbols
0

% :LatexSympyOp symbols_reset
0

% :LatexSympyOp geometry
Point(0, 0)

% :LatexSympyOp intersect
% (select both lines together)
Line(Point(0, 0), Point(1, 1))
Line(Point(0, 1), Point(1, 0))

% :LatexSympyOp tangent
% (select both lines together)
Circle(Point(0, 0), 5)
Line(Point(5, 0), Point(5, 1))

% :LatexSympyOp similar
% (select both lines together)
Polygon(Point(0, 0), Point(1, 0), Point(0, 1))
Polygon(Point(0, 0), Point(2, 0), Point(0, 2))

% :LatexSympyOp units simplify
2*meter + 3*meter

% :LatexSympyOp units convert kilometer/hour
10*meter/second

% :LatexSympyOp mechanics euler_lagrange q(t)
Derivative(q(t), t)^2/2 - q(t)^2/2

% :LatexSympyOp quantum dagger
A*B

% :LatexSympyOp quantum commutator B
A

% :LatexSympyOp optics lens focal_length=2 u=3
0

% :LatexSympyOp optics mirror focal_length=2 u=3
0

% :LatexSympyOp optics refraction 1 1 2
0

% :LatexSympyOp pauli simplify
Pauli(1)*Pauli(1)

% :LatexSympyOp dist normal X 0 1
0

% :LatexSympyOp p
X > 0

% :LatexSympyOp e
X

% :LatexSympyOp var
X

% :LatexSympyOp density
X

% :LatexSympyRepeat
% (run this right after any LatexSympyOp / alias command)
x^3 + x

\section*{Alias commands}

% :LatexSympySolve x
x^2 - 9 = 0

% :LatexSympyDiff x 3
x^5

% :LatexSympyIntegrate x 0 2
x^2

% :LatexSympyDet
\begin{bmatrix} 2 & 0 \\ 0 & 3 \end{bmatrix}

% :LatexSympyInv
\begin{bmatrix} 2 & 1 \\ 1 & 1 \end{bmatrix}

\section*{Utility commands (no selection required)}

% :LatexSympyStatus
% :LatexSympyStart
% :LatexSympyStop
% :LatexSympyRestart
% :LatexSympyVariances
% :LatexSympyReset
% :LatexSympyToggleComplex
% :LatexSympyPick

\section*{Python eval (opt-in only)}

% Enable first:
% :lua require("latex_sympy").setup({ enable_python_eval = true })
%
% Then select the line below and run:
% :LatexSympyPython
1 + 1

\end{document}
